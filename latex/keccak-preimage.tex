\documentclass{article}
\usepackage[utf8]{inputenc}
\usepackage{hyperref}
\title{keccak preimage}
\usepackage{graphicx}
\usepackage{tikz}
\usepackage{tikz,graphicx}
\usepackage[absolute,overlay]{textpos} 
\usetikzlibrary{decorations.pathmorphing,matrix,decorations.pathreplacing,arrows,decorations.markings}
\usetikzlibrary{fit,calc,shapes,arrows,positioning,shadings,backgrounds,patterns,tikzmark,matrix,spy}
\usetikzlibrary{decorations.markings}

\usepackage[T1]{fontenc}
\begin{document}

\maketitle

\section{Introduction}

The observations based on the following paper : \textbf{Linear Structures: Applications to Cryptanalysis of Round-Reduced Keccak}.

\subsection{2R Keccak-512}

See Fig. 8, for each guess :
we set 
\[
    A[0, 1] = A[0, 0] \oplus \alpha_{0}
\]
\[
    A[2, 1] = A[2, 0] \oplus \alpha_{2}
\]
with $\alpha_0$ and $\alpha_2$ as random constants

Since $A[0,0]$ and $A[2,0]$ have 128 bits. So we have a complexity gain over bruteforce of $2^{128}$. Hence the time complexity $= 2^{512 - 128} = 2^{384}$.

Input degree of freedom : 
\begin{enumerate}
    \item 64 bits from $A[0, 0]$ , 64 bits from $A[2,0]$.
    \item (320 -1) bits from white lanes
    \item 128 bits from $\alpha_0, \alpha_2$
\end{enumerate}

This sums upto 575 bits larger than required 512 bits.


\end{document}